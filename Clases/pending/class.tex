\documentclass{article}
\usepackage{graphicx}
\usepackage{ragged2e}

\begin{document}
\centering
{\Large La Comunicación }
\vspace{1cm}

\textbf{Comunicación Humana}

mensaje origen - transmisor - medio - receptor - mensaje destino \vspace{0.5cm}

\justify
Establecimiento de reglas:
\begin{itemize}
    \item emisor y recept identificados
    \item metodo de Comunicación
    \item idioma y grmaática común
    \item velocidad y momento de entrega
    \item requisitos de confirmación o acuse de recibo
\end{itemize}
\vspace{0.5cm}

\centering
\textbf{Codificación del mensaje}

origen - codificador - transmisor - medio(canal) - receptor - decodificador -
destino del mensaje \vspace{0.5cm}

\justify
para la transmisión de un msg se tiene a veces un tamaño que supera las restricciones
por eso es necesario segmentar el mensaje largo e fragmentos que cumpla con requisitos
de tamaño minimo y máximo.
\vspace{0.5cm}

Temporización del mensaje
\begin{itemize}
    \item metodo de acceso
    \item control de flujo: Manera del emisor de evitar saturar al receptor
    \item tiempo de espera para la respuesta:
\end{itemize}

\vspace{0.5cm}

Opciones de entrega del mensaje

\emph{--Métodos de difusión de un mensaje--}
\begin{itemize}
    \item Unicast: Uno a uno
    \item Multicast: Comunicación a un subgrupo
    \item Broadcast: Uno a muchos
\end{itemize}
\newpage

Protocolos de red
\begin{itemize}
    \item Formato o estructuración dle mensaje.
    \item Proceso por el cual los dispositivos de red
          comparten info sobre las rutas con otras redes
    \item Cómo y cuando se transmiten mensajes de error y del sistema entre los
          dispositivos
    \item La configuración y la terminación de sesiones de transferencia de datos
\end{itemize}
\vspace{0.5cm}

Interacción de Protocolos
\begin{itemize}
    \item http
    \item Tcp/ip
\end{itemize}
\vspace{0.5cm}

Organismos de estandarización
\begin{itemize}
    \item ISOC
    \item IAB
    \item identificados
    \item IEEE
    \item International Organization for Standarization (ISO)
\end{itemize}
\end{document}