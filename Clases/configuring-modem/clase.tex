\documentclass{article}
\usepackage{graphicx}

\begin{document}
\section{Getting started with the Cisco configuration}

\centering
{ \Large \textbf{Sistemas operativos}}
\vspace{1cm}

\emph{Cisco IOS (Internetwork Operating System): A family of proprietary network operating systems developed by Cisco Systems. }
\vspace{1cm}

Los SO de nuestros dispositivos intermedios proporcionan opciones para
\begin{itemize}
    \item Configurar interfaces
    \item Habilitar funciones de enrutamiento
\end{itemize}
\vspace{1cm}

Todos los dispositivos de red viene con un IOS predetermiado
\vspace{1cm}

\textbf{Ubicación de cisco IOS}

El IOS se almacena en flash
\begin{itemize}
    \item Almacenamiento no volátil: o se pierde cuando se corta alimentación
    \item Se puede modificar o sobreescribir
    \item El IOS se copia de la memoria flash a la RAM
    \item La cantidad de memoria flash y RAM determina qué IOS se puede utilizar
\end{itemize}

\vspace{1cm}

Funciones de IOS
\begin{itemize}
    \item Seguridad
    \item Administración recursos
    \item etc
\end{itemize}

\vspace{1cm}

\textbf{Metodos para Configurar}

\begin{itemize}
    \item Consola
    \item TElnet o SSH (conexión remota)
    \item Puerto auxiliar (puerto fisico)
\end{itemize}

\vspace{1cm}

\textbf{Modos de funcionamiento de Cisco IOS}

\begin{enumerate}
    \item Modo usuario: Uso limitado de comandos
    \item Modo exec privilegiado: Debug commands, reload, configure, etc.
    \item Configuración global (super sudo): Interfaces, router config, line commands
          router(password, login, modem commands)
\end{enumerate}

\vspace{1cm}

\textbf{Some commands}
\begin{itemize}
    \item \(\)?: Gives some commands
    \item enable: Enables mode exec privilegiado
    \item configure terminal: enables super sudo user
    \item Show running-config: The one we just made
    \item show startup-config: Show saved config fpr startup
    \item hostname <name>: change router name
    \item write: save config
\end{itemize}

\end{document}