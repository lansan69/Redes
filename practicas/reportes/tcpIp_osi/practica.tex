\documentclass[a4paper,11pt]{article}
\usepackage[utf8]{inputenc}
\usepackage[backend=biber]{biblatex}
\addbibresource{referencias.bib}
\usepackage[a4paper, top=3cm, bottom=3cm, left=3cm, right=3cm]{geometry}
\usepackage[T1]{fontenc}
\usepackage{multirow}
\usepackage{booktabs}
\usepackage{graphicx}
\usepackage{rotating}
\usepackage{float}
\usepackage{fancyhdr}
\usepackage{setspace}
\usepackage{listings}
\usepackage{ragged2e} % for \justifying
\usepackage{enumitem} % better lists
\usepackage{hyperref} % hyperlinks
\usepackage{xcolor}
\usepackage{amsmath,amssymb,textcomp}
\usepackage{multicol}
\usepackage{tikz}
\usepackage{pdflscape} % en el preámbulo

% Header & footer
\pagestyle{fancy}
\fancyhf{}
\lhead{\footnotesize EQUIPO 1 }
\rhead{\footnotesize 5CV3} 
\cfoot{\footnotesize \thepage}
\setlength{\parindent}{0in}

\begin{document}
\vspace{0.4cm}

\begin{titlepage}
    \centering
    {\bfseries\LARGE Escuela Superior de Cómputo \par}
    \vspace{1cm}
    {\scshape\Large REDES DE COMPUTADORAS \par}
    \vspace{5cm}
    {\scshape\Huge \textbf{Práctica de laboratorio:} \par}
    {\scshape\Huge Use of the TCP/IP Protocols and the OSI Model in Packet Tracer \par}
    \vfill
    {\Large \textbf{Por el equipo 1:} \par}
    {\Large Barrera Puente Eric Alejandro \par}
    {\Large Diaz Villegas Ramón Alexis \par}
    {\Large Sánchez Gómez Alan Iván \par}
    %comentario de prueba para cambios en github
    \vspace{1cm}
    {\Large \textbf{Grupo: 5CV3} \par}
    \vfill
    {\Large 24 de Septiembre, 2025 \par}
\end{titlepage}

\tableofcontents %índice
\newpage

\listoffigures
\newpage

\justify
\section{Objetivos de aprendizaje}
\begin{itemize}
    \item Explorar cómo Packet Tracer utiliza el modelo OSI y los protocolos TCP/IP.
    \item Examinar el procesamiento y contenido de los paquetes.
\end{itemize}

\section{Desarrollo de la práctica}

\subsection{Introducción}
En el modo de simulación de Packet Tracer, se puede observar información detallada sobre los paquetes y cómo los dispositivos de red los procesan. Los protocolos TCP/IP comunes modelados en Packet Tracer incluyen DNS, HTTP, TFTP, DHCP, Telnet, TCP, UDP, ICMP e IP. El término \textbf{PDU (Protocol Data Unit)} es una descripción genérica de los segmentos en la capa de transporte, paquetes en la capa de red y tramas en la capa de enlace de datos.

\subsection{Tarea 1: Explorar la interfaz de Packet Tracer}

\subsubsection{Paso 1: Revisar los archivos de ayuda y tutoriales}
\begin{enumerate}[label=\alph*.]
    \item En el menú desplegable, elegir \textbf{Help $\rightarrow$ Contents}.
    \item En la página que se abre, seleccionar \textbf{Operating Modes $\rightarrow$ Simulation Mode}.
    \item Leer la información sobre el modo de simulación si no se está familiarizado.
\end{enumerate}
% \includegraphics[width=0.8\textwidth]{ruta_a_captura}

\subsubsection{Paso 2: Cambiar de Realtime a Simulation Mode}
\begin{enumerate}[label=\alph*.]
    \item Ubicar el interruptor en la esquina inferior derecha de la interfaz de PT.
    \item Hacer clic en el ícono de \textbf{Simulation Mode} para cambiar de Realtime.
    \item Notar que en Simulation Mode los paquetes se muestran como sobres animados y el tiempo es controlado por eventos.
\end{enumerate}
% \includegraphics[width=0.8\textwidth]{ruta_a_captura}

\subsection{Tarea 2: Examinar el contenido y procesamiento de paquetes}

\subsubsection{Paso 1: Crear un paquete y acceder a la ventana de información PDU}
\begin{enumerate}[label=\alph*.]
    \item Seleccionar el PC Web Client y abrir la pestaña \textbf{Desktop}.
    \item Abrir el \textbf{Web Browser} y escribir la IP del servidor web (\texttt{192.168.1.254}).
    \item Hacer clic en \textbf{Go} para enviar la solicitud HTTP.
    \item Minimizar la ventana del Web Client y usar el botón \textbf{Capture/Forward} para mostrar eventos de red.
    \item Hacer clic en el cuadrado de información (Info) del primer paquete en la lista de eventos.
\end{enumerate}
% \includegraphics[width=0.8\textwidth]{ruta_a_captura}

\subsubsection{Paso 2: Investigar los algoritmos de los dispositivos en la vista OSI}
\begin{enumerate}[label=\alph*.]
    \item Abrir la ventana de información PDU haciendo clic en el sobre del paquete o el cuadrado Info.
    \item Observar cómo la solicitud HTTP (capa 7) se encapsula en capas 4, 3, 2 y 1.
    \item Explorar cada capa para ver el algoritmo del dispositivo.
\end{enumerate}
% \includegraphics[width=0.8\textwidth]{ruta_a_captura}

\subsubsection{Paso 3: PDU entrantes y salientes}
\begin{enumerate}[label=\alph*.]
    \item En la ventana PDU, seleccionar la pestaña \textbf{Outbound PDU Details}.
    \item Observar el proceso de encapsulación: HTTP $\rightarrow$ TCP $\rightarrow$ IP $\rightarrow$ Ethernet $\rightarrow$ bits.
    \item Notar las diferencias entre PDU salientes y entrantes según el dispositivo.
\end{enumerate}
% \includegraphics[width=0.8\textwidth]{ruta_a_captura}

\subsubsection{Paso 4: Rastreo de paquetes: animación del flujo de paquetes}
\begin{enumerate}[label=\alph*.]
    \item Usar \textbf{Capture/Forward} para capturar eventos paso a paso.
    \item Observar los paquetes HTTP, TCP y ARP mientras atraviesan la red.
    \item Abrir la ventana PDU en cualquier momento para inspeccionar los paquetes.
    \item Repetir la animación para predecir e investigar el flujo de paquetes.
\end{enumerate}
% \includegraphics[width=0.8\textwidth]{ruta_a_captura}

\section{Conclusiones}
\begin{itemize}
    \item \textbf{Barrera Puente Eric Alejandro:}
    

    \item \textbf{Díaz Villegas Ramón Alexis:}


    \item \textbf{Sánchez Gómez Alan Iván:}
\end{itemize}

\end{document}